The present coursework aims to conduct, under the knowledge of Text Mining and related areas, an
investigation about the impacts of clustering algorithms in the development of the flexible
organization of textual documents. In the beginning of this coursework, it is discussed the
relevance and the choices on  preprocessing stage, that is prior to clustering proper, besides, the
validation criteria of clustering during the post processing stage, namely, fuzzy silhouette. To
contextualize the piratical of the proposed flexible textual organization, it is remarked along
chapters the intrinsic challenges of text organization, for example, the  higher computational costs
problem in seeking semantic similarities degrees on sparse matrices value-attribute matrices, as
well as possible ways to mitigate or reduce the negative effects of these difficult on the process
meaning attribution groups that are outcome from clustering though relevant descriptors extraction
The strategy of the present course work consists in an approach of flexible and hybrid document
organization, mixing the benefits of the known fuzzy partition interpretation and possibilistic one
on clustering method of Possibilistic C-Means e Possibilistic Fuzzy C-Means (PFCM). As an output of
here developed experiments, it were proposed the Possibilistic Description Comes Last (PDCL) and
Mixed - Possibilistic Fuzzy Description Comes Last (PFDCL) descriptors' extraction methods.
Both were confirmed by experimental evidences and analysis that are subjective to methods adjusts
for the flexible organization of documents collaborating for original discoveries on the state of
arts. The results of the present research, after all, stimulate new implementations whose execution
may be executed on future works.
