The present coursework aims to conduct, under the knowledge of Text Mining and related areas, an
investigation about the impacts of clustering algorithms in the development of the flexible
organization of textual documents. In the beginning of this coursework, it is discussed the
relevance and the choices on  preprocessing stage of documents that will be organized; the theory
and methods used to clustering documents for find useful patterns from these documents and how to
validate the resulted organization. For present the flexible document organization proposed process,
the challenges that are inherent to it, also are elucidated from related work. 
The conducted research to develop this coursework has the main contribution a hybrid  strategy for
flexible organize a collection of documents, gathering the benefits from a new approach to right
interpret the fuzzy and possibilistic partitions produced in Possibilistic C-Means (PCM) and
Possibilistic Fuzzy C-Means (PFCM) clustering methods.
As an output of here developed experiments, it were proposed the Possibilistic Description Comes
Last (PDCL) and Mixed - Possibilistic Fuzzy Description Comes Last (PFDCL) descriptors' extraction
methods.  Both were confirmed by experimental evidences and analysis that are subjective to methods
adjusts for the flexible organization of documents collaborating for original discoveries on the
state of arts. The results of the present research, after all, stimulate new implementations whose
execution may be executed on future works.
